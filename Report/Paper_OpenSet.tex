% CVPR 2022 Paper Template
% based on the CVPR template provided by Ming-Ming Cheng (https://github.com/MCG-NKU/CVPR_Template)
% modified and extended by Stefan Roth (stefan.roth@NOSPAMtu-darmstadt.de)

\documentclass[10pt,twocolumn,letterpaper]{article}

%%%%%%%%% PAPER TYPE  - PLEASE UPDATE FOR FINAL VERSION
%\usepackage[review]{cvpr}      % To produce the REVIEW version
\usepackage{cvpr}              % To produce the CAMERA-READY version
%\usepackage[pagenumbers]{cvpr} % To force page numbers, e.g. for an arXiv version

% Include other packages here, before hyperref.
\usepackage{graphicx}
\usepackage{amsmath}
\usepackage{amssymb}
\usepackage{booktabs}


% It is strongly recommended to use hyperref, especially for the review version.
% hyperref with option pagebackref eases the reviewers' job.
% Please disable hyperref *only* if you encounter grave issues, e.g. with the
% file validation for the camera-ready version.
%
% If you comment hyperref and then uncomment it, you should delete
% ReviewTempalte.aux before re-running LaTeX.
% (Or just hit 'q' on the first LaTeX run, let it finish, and you
%  should be clear).
\usepackage[pagebackref,breaklinks,colorlinks]{hyperref}


% Support for easy cross-referencing
\usepackage[capitalize]{cleveref}
\crefname{section}{Sec.}{Secs.}
\Crefname{section}{Section}{Sections}
\Crefname{table}{Table}{Tables}
\crefname{table}{Tab.}{Tabs.}


%%%%%%%%% PAPER ID  - PLEASE UPDATE
\def\cvprPaperID{*****} % *** Enter the CVPR Paper ID here
\def\confName{CVPR}
\def\confYear{2022}


\begin{document}

%%%%%%%%% TITLE - PLEASE UPDATE
\title{Open-Set Domain Adaptation through Self-Supervision}

\author{Daniele Rege Cambrin, Kylie Bedwell, Tommaso Natta, Ehsan Ansari Nejad\\
Politecnico di Torino\\
Corso Duca degli Abruzzi, 24\\
10129 Torino, ITALY\\
{\tt\small s290144@studenti.polito.it, s287581@studenti.polito.it} \\
{\tt\small s282478@studenti.polito.it, s288903@studenti.polito.it}
% For a paper whose authors are all at the same institution,
% omit the following lines up until the closing ``}''.
% Additional authors and addresses can be added with ``\and'',
% just like the second author.
% To save space, use either the email address or home page, not both
}
\maketitle

%%%%%%%%% ABSTRACT
\begin{abstract}
   - sentence describing the problem
 - sentence describing our proposed method
 - sentence summarising the results
 - sentence about the variations
 - sentence about the variation results
\end{abstract}

%%%%%%%%% BODY TEXT
\section{Introduction}
\label{sec:intro}

In the computer vision research area large amounts of unlabeled data are available, however the cost of labelling this data is high ~\cite{Csurka2017}. Domain adaptation is one technique that can be used to exploit the unlabeled data by first training a model on labeled data from a different but similar domain (the \textit{source} domain), and then applying this model to the unlabeled data ( the \textit{target} domain). This technique assumes the distribution of both source and target domans are similar and describe the same class labels, also known as the \textit{closed-set} scenario~\cite{Bucci2020}. When applied to real-word scenarios however it is possible that the target domain includes previously unseen classes, known as the \textit{open-set} scenario. These extra class labels in the target domain will cause performance degradation of the classification model and should be identified and isolated. The problem thus consists of two steps: first separating the target domain into known and unknown samples; then conducting domain alignment between the source domain and the known samples of the target domain. 

Self-supervised learning can be used to separate the known class samples in the target domain from the unknown samples. Self-supervised learning involves using known ...


- Introduce the problem
- state our focus
- the network used
- the dataset used
- introduce the variations we will implement

%------------------------------------------------------------------------
\section{Related Work}
\label{sec:relatedWork}

- what's been done in the past on this topic
 - Summarise each of the cited references
- Cite some papers about the variations we did
 - other variations which might be interesting (but that we didnt do) - if space
- 

%------------------------------------------------------------------------
\section{Method}
\label{sec:method}

- describe in detail the method used
- include the diagrams here and explain them
- explain the evaluation parameters that will be used

%------------------------------------------------------------------------
\section{Experiments}
\label{sec:experiments}

- start with parameter tuning of learning rate and epochs, include graph to justify decision
- ablation study: hyperparameter tuning of weights and threshold value, include graphs or tables of values


%------------------------------------------------------------------------
\section{Variations}
\label{sec:variations}

- describe each of the variations
- present results of the performance of the model and compare with the baseline for each of the variations

%------------------------------------------------------------------------
\section{Conclusions}
\label{sec:conclusion}

- summarise the results
- add future recommendations

\subsection{Acknowledgements}
The authors would like to thank Silvia Bucci for her assistance and guidance in completing this study.


%%%%%%%%% REFERENCES
{\small
\bibliographystyle{ieee_fullname}
\bibliography{egbib}
}

\end{document}
